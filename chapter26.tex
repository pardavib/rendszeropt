
\setcounter{chapter}{26}
%----------------------------------------------------------------------------
\chapter*{26. Tétel}
%----------------------------------------------------------------------------

\textbf{Témakörök:} Kirchoff eredményeinek általánosítása: szükséges feltétel tetszőleges lineáris sok-kapukat is tartalmazó hálózatok egyértelmű megoldhatóságára. Villamos hálózatok duálisa.

\noindent\hrulefill

\section*{Többpolusú elemek}
n-kapu:
\begin{itemize}
\item $n$ póluson kapcsolódik
\item $A_{u}+B_{i}=0$, ahol $(A|B)$ $n$-rangú, $n\times 2n$ mátrix
\end{itemize}

\noindent
Transzformátor:
\[ \left( \begin{array}{cccc}
k & -1 & 0 & 0 \\
0 & 0 & 1 & k \end{array} \right)\] 

\noindent
Girátor:
\[ \left( \begin{array}{cccc}
1 & 0 & 0 & -R \\
0 & 1 & R & 0 \end{array} \right)\] 

\section*{Szükséges feltétel}
\begin{theo}
$H$ egy $U,I$ forrásokból és lineáris sokkapukból álló hálózat, melyben:
\begin{itemize}
\item $E_{u}$ körmentes, $E_{i}$ vágásmentes részgráf,
\item helyettesítsük a feszültségforrásokat rövidzárral, az áramforrásokat szakadással.
\end{itemize}
Ekkor $H^{'}$ egyértelműen megoldható $\Leftrightarrow$ $H$ egyértelműen megoldható.
\end{theo}

\noindent
Jelölje:
\begin{itemize}
\item $G:=$ a $H^{'}$ gráfját
\item $A_{U}, A_{I}:=$ sokkapuk feszültségei és áramai
\end{itemize}

\noindent
Ekkor:
\begin{itemize}
\item $A=A_{U}\cup A_{I}$ halmazon $p$ lineárisan független egyenlet
\item $N=p\times 2p$ méretű együtthatómátrixot határoz meg.
\end{itemize}

\begin{theo}
$H^{'}$ egyértelmű megoldhatóságának szükséges feltétele, hogy $A$ kettéosztható $A_{g}\cup A_{a}$-ra úgy, hogy:
\begin{itemize}
\item $A_{g}\cap A_{u}$ vágásmentes $G$-ben és $A_{g}\cap A_{i}$ körmentes $G$-ben,
\item valamint $N$ mátrix $A_{a}$-nak megfelelő oszlopai lineárisan függetlenek.
\end{itemize}
\end{theo}

\section*{Dualitás}
\begin{itemize}
\item Ha $H_{1}$ hálózat kétpólusú alkatrészekből áll,
\item $G_{1}$ kapcsolás gráfja síkbarajzolható,
\end{itemize}
akkor a hálózatot leíró egyenletek ugyanazok lesznek, ha:
\begin{itemize}
\item $G_{2}:=G_{1}$ duálisát tekintjük,
\item és $U,I$ betűket felcseréljük.
\end{itemize}

\subsection*{Alkatrészek megfelelői}

\begin{center}
\begin{tabular}{ l | c }
  \textbf{Eredeti} & \textbf{Duális} \\
  élek & élek \\
  körök & vágások \\
  vágások & körök \\
  feszültség & áram \\
  áram & feszültség \\
  ellenállás (alkatrész) & ellenállás \\
  ellenállás (fizikai mennyiség) & vezetés \\
  feszültségforrás & áramforrás \\
  áramforrás & feszültségforrás \\
  tekercs & kondenzátor \\
  kondenzátor & tekercs \\
\end{tabular}
\end{center}

Ha a $H_{1}$-beli alkatrészeket a fenti feltételek mellett, a szótár alapján helyettesítjük, akkor az így kapott $H_{2}$ hálózat és az eredeti $H_{1}$ alkatrészeinek feszültségeit és áramait leíró teljes egyenletrendszerek formailag azonosak lesznek, csak az $U,I$ betűket kell felcserélni.\\
\newline
Lineáris alkatrészek esetén a betűcserés és matroidelméleti dualitás eltérő eredményt adhat. (Például egy erősítő feszültségvezérelt feszültségforrás marad a matroid duálisát véve.)