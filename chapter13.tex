
\setcounter{chapter}{13}
%----------------------------------------------------------------------------
\chapter*{13. Tétel}
%----------------------------------------------------------------------------

\textbf{Témakörök:} A $k$-matroid partíciós probléma, ennek algoritmikus megoldása. A 2-matroid-metszet feladat visszavezetése matroid partíciós problémára.

\noindent\hrulefill

\section*{k-matroid partíciós probléma (k-MPP vagy {MPP\textsubscript{k}})}
Adott: $k$ db matroid: $M_{i}=(E,F_{i}), i=1,2,\cdots ,k$\\
Kérdés: a matroidok összege a szabadmatroidot adja-e, vagyis $E$ előáll-e $E_{1}\cap\cdots\cap E_{k}$ alakban úgy, hogy $E_{i}\in F_{i} \forall i$-re.\\
Feltehető, hogy az $E_{i}$ halmazok diszjunktak, ezért hívják a feladatott MPP-nak.

\subsection*{Bonyolultság}
\begin{itemize}
\item {MPP\textsubscript{k}} NP-beli, mert tanú rá egy partícionálás és a tanú polinom (lineáris) iidőben ellenőrizhető.
\item {MPP\textsubscript{k}} co-NP-beli, mert tanú rá egy $X\subseteq E$ halmaz, ami biztosan összefüggő az összegben, azaz $|X|>\sum r_{i}(X)$.
\item {MPP\textsubscript{k}} P-beli.
\end{itemize}

\subsection*{Algoritmus}
Induljunk ki $\forall i E_{i}=\emptyset$ állapotból. Ekkor $E_{i}\in F_{i}$.\\
Az $E_{i}$ halmazokat addig bővítjük, amíg az uniójuk $E$ nem lesz, vagy ha nem bővíthető, mutatunk egy $X$ tanút.\\
\newline
A bővítéshez bevezetünk egy $n+k$ pontú irányított segédgráfot, melynek:
\begin{itemize}
\item csúcsai $E$ elemei $\cup\lbrace p_{1},\cdots ,p_{k}\rbrace$. $p_{i}$ az $E_{i}$ partíció segédpontja
\item $(x\rightarrow p_{i})\in E(G)$, ha $x\notin E_{i}$ és $E_{i}\cup\lbrace x\rbrace\in F_{i}$\\
Az ilyen típusú élek azt jelképezik, hogy az $E_{i}$ partícióba felvehető $x$ a függetlenség megsértése nélkül.
\item $(x\rightarrow y)\in E(G)$, ha $\exists i: x\notin E_{i}$, $y\in E_{i}$, $E_{i}\cup\lbrace x\rbrace\notin F_{i}$, de $E_{i}\cup\lbrace x\rbrace-\lbrace y\rbrace\in F_{i}$\\
Az ilyen típusú élek azt jelentik, hogy az $E_{i}$ partícióban az $y$ elem kicserélhető $x$-re a függetlenség megsértése nélkül.
\end{itemize}

\subsection*{Lépések}
\begin{itemize}
\item 1. Megkeressük a legrövidebb irányított utat $(E-\cup E_{i})$-ből $\lbrace p_{1},\cdots ,p_{k}\rbrace$-ba.
\item 2. Ha van ilyen út, javítunk az út mentén, azaz végrehajtjuk a cseréket. $\cup E_{i}$ mérete 1-gyel nő.\\
Azért kell a legrövidebb úton végigmenni, mert különben nem garantált, hogy a cserék során nem sérül a partíciók függetlensége.
\item 3. Különben STOP, nemleges a válasz, és a tanú az $(E-\cup E_{i}$)-ből irányított úton elérhető pontok halmaza.
\end{itemize}

\section*{2-matroid-metszet probléma}
\subsection*{{MMP\textsubscript{k}} (emlék)}
Adott: $k$ db matroid: $M_{i}=(E,F_{i}), i=1,2,\cdots ,k$ és egy $p$ egész szám\\
Kérdés: létezik-e $F_{i}$-knek legalább $p$ méretű közös elemük?
\begin{theo} {MMP\textsubscript{2}} $\in P$ \end{theo}
\begin{defn} [Csonkolt]
$M=(E,F)$ csonkoltja $M^{'}=(E,F^{'})$, ahol $F^{'}$ az $F$ elemeit tartalmazza a bázisok kivételével. A matroid rangja ettől 1-gyel csökken.
\end{defn}