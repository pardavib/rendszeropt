
\setcounter{chapter}{24}
%----------------------------------------------------------------------------
\chapter*{24. Tétel}
%----------------------------------------------------------------------------

\textbf{Témakörök:} Kirchoff tételei a klasszikus villamos hálózatok analízisére.

\noindent\hrulefill

Kirchoff 1847-ben írt cikkében több eredményének megfogalmazásához és bizonyításához használt fel gráfelméleti eredményeket.\\
\newline
Villamos hálózatok összekapcsolása $\rightarrow$ gráfokkal modellezhető:
\begin{itemize}
\item az alkatrészek a gráfok éleinek felelnek meg,
\item az elmélet kétpólusú alkatrészeket tartalmazó hálózatokra alkalmazható közvetlenül.
\end{itemize}

\section*{Alapvető ismeretek}
Egy villamos hálózat kizárólag
\begin{itemize}
\item $U=R*I$ egyenletű ellenállásokat,
\item $\lbrace U$ adott, $I$ tetszőleges$\rbrace$ idealizált feszültségforrásokat,
\item $\lbrace I$ adott, $U$ tetszőleges$\rbrace$ idealizált áramforrásokat tartalmaz.
\end{itemize}
Továbbá:
\begin{itemize}
\item[•] összekapcsolás: $G=(V,E)$ gráf,
\item[•] $E=E_{R}\cup E_{U}\cup E_{I}$ (ellenállások, feszültség- és áramforrások halmaza),
\item[•] élek irány: az alkatrész "mérőiránya".
\end{itemize}

\subsection*{Hálózatanalízis alapfeladata}
Ismert: ellenállások, feszültségforrások feszültségei, áramforrások áramai.\\
Cél: meghatározni a többi adatot.

\subsection*{Ohm-törvény}
$R=\frac{U}{I}$

\subsection*{Kirchoff I. (Huroktörvény)}
A hálózat gráfjának bármely köre mentén az alkatrészek feszültségeinek előjeles összege zérus.

\subsection*{Kirchoff II. (Csomóponti törvény)}
A hálózat gráfjának bármely vágása mentén az alkatrészek áramainak előjeles összege zérus.\\
\newline
\textbf{Megjegyzések:}
\begin{itemize}
\item az alapfeladat egy lineáris egyenletrendszer megoldása lesz
\item kétszeresen összefüggő gráfokban a vágásokkal ekvivalensen értelmezhető csillagokra is a csomóponti törvény
\item $U=0$: speciális feszültségforrás, rövidzár
\item $I=0$: speciális áramforrás, szakadás
\end{itemize}

\begin{lem}
Egy ilyen hálózat egyértelmű megoldhatóságának szükséges feltétele:
\begin{itemize}
\item $E_{U}$ élhalmaz körmentes,
\item $E_{I}$ élhalmaz vágásmentes részgráfot határozzon meg.
\end{itemize}
\end{lem}

\begin{theo}
A fenti feltétel szükséges és elégséges lesz, ha a hálózat:
\begin{itemize}
\item feszültségforrásokat,
\item áramforrásokat,
\item és \textbf{pozitív} ellenállásokat tartalmaz.
\end{itemize}
\end{theo}

\begin{theo}
A fenti feltételek mellett az összes ismeretlen közös nevezője: 
$detM=\sum\limits_{F\in F}\prod\limits_{i\notin F} R_{i}$\\
ahol $F$ a gráf azon részgráfjainak halmaza, melyek:
\begin{itemize}
\item $E_{I}$ egyetlen élét sem tartalmazzák,
\item és $E_{U}$-valegyesítve fát alkotnak.
\end{itemize}
Ezek az úgynevezett "topológiai formulák".
\end{theo}

\subsection*{Egyéb kétpólusú alkatrészek}
Tekercs: $U=L*\frac{dI}{dt}$ ($L$: induktivitás)\\
Kondenzátor: $I=C*\frac{dU}{dt}$ ($C$: kapacitás)\\
(Feszültség- és áramforrás függhet az időtől.) $\Rightarrow$ Nem szakadás/rövidzár.

\subsection*{További tételek}
\begin{theo} [1]
Feszültség- és áramforrásokat, pozitív ellenállásokat, tekercseket és kondenzátorokat tartalmazó hálózat egyértelmű megoldhatóságának szükséges és elégséges feltétele:
\begin{itemize}
\item $E_{U}$ körmentes,
\item $E_{I}$ vágásmentes részgráfot határozzon meg.
\end{itemize}
\end{theo}

\begin{theo} [2]
Ha még $E_{U}\cup E_{C}$ is körmentes, valamint $E_{I}\cup E_{L}$ vágásmentes, akkor a differenciálegyenlet $|E_{C}|+|E_{L}|$ rendű, és függetlenül előírhatók a kondenzátorok kezdeti feszültségei és a tekercsek kezdeti áramai.
\end{theo}

\noindent
Ha (1) tétel teljesül, legyen $F$ gráf olyan, hogy:
\begin{itemize}
\item tartalmazza $E_{U}$ összes elemét,
\item $E_{C}$-ből a lehető legtöbb elemet ($X_{C}$), hogy $E_{U}\cup X_{C}$ is körmentes legyen,
\item $E_{I}$ egyetlen elemét sem tartalmazza,
\item $E_{L}$-ből a lehető legkevesebb elemet, hogy a kimaradó $X_{L}$-ekre $E_{I}\cup X_{L}$ is vágásmentes legyen,
\end{itemize}
akkor a differenciálegyenlet $|X_{C}|+|X_{L}|$ rendű, továbbá az $X_{C}$ kezdeti feszültség és $X_{L}$ kezdeti áramerősség előírható függetlenül.



