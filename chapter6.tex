

%----------------------------------------------------------------------------
\chapter*{6. Tétel}
%----------------------------------------------------------------------------

\textbf{Témakörök:} A lineáris és egészértékű programozás alkalmazása páros gráfokra és intervallumrendszerekre: Egerváry tétele, intervallumrendszerek egyenletes színezése.

\noindent\hrulefill

\begin{defn} [Illeszkedési mátrix] Legyen $n$ pontú gráfnak $e$ éle és definiáljuk az $n\times e$ méretű $B(G)=b_{ij}$ mátrix elemeit, hogy: \end{defn}

\begin{theo} Minden irányított gráf illeszkedési mátrixa TU. \end{theo}

\begin{prf} [Teljes indukció] 
Válasszunk $M$ $k\times k$-as részmátrixot.
\begin{itemize}
	\item ha $k=1$, akkor nyilvánvaló az állítás, hisz minden elem $0$ vagy $\pm 1$
	\item ha $k\geq 2$ és:
	\begin{itemize}
		\item $M$-nek van olyan oszlopa, melyben legfeljebb egy nemnulla elem van, akkor fejtsük ki $detM$-et eszerint az oszlop szerint, ekkor az indukciós feltétel szerint készen vagyunk.
		\item egyébként minden oszlopban egy $+1$ és egy $-1$ elem van, ekkor $M$ sorainak összege nullvektor, a determináns $0$.
	\end{itemize}
\end{itemize}
\end{prf}

\begin{theo} Páros gráf illeszkedési mátrixa TU. \end{theo}

\begin{prf}
Irányítsuk $G(A,B,E)$ páros gráf éleit úgy, hogy minden él A-ból B-be mutasson. Ekkor az előző tétel szerint $B(G)$ TU. A B-hez tartozó sorokat szorozzuk $-1$-gyel, de ez nem változtat TU tulajdonságon.
\end{prf}

%----------------------------------------------------------------------------
\section{A dolgozat kimérete}
%----------------------------------------------------------------------------
A minimális 50, az optimális kiméret 60-70 oldal (függelékkel együtt). A bírálók és a záróvizsga bizottság sem szereti kifejezetten a túl hosszú dolgozatokat, így a bruttó 90 oldalt már nem érdemes túlszárnyalni. Egyébként függetlenül a dolgozat kiméretétől, ha a dolgozat nem érdekfeszítő, akkor az olvasó már az elején a végét fogja várni. Érdemes zárt, önmagában is érthető művet alkotni.

%----------------------------------------------------------------------------
\section{A dolgozat nyelve}
%----------------------------------------------------------------------------
Mivel Magyarországon a hivatalos nyelv a magyar, ezért alapértelmezésben magyarul kell megírni a dolgozatot. Aki külföldi posztgraduális képzésben akar részt venni, nemzetközi szintű tudományos kutatást szeretne végezni, vagy multinacionális cégnél akar elhelyezkedni, annak célszerű angolul megírnia diplomadolgozatát. Mielőtt a hallgató az angol nyelvű verzió mellett dönt, erősen ajánlott mérlegelni, hogy ez mennyi többletmunkát fog a hallgatónak jelenteni fogalmazás és nyelvhelyesség terén, valamint - nem utolsó sorban - hogy ez mennyi többletmunkát fog jelenteni a konzulens illetve bíráló számára. Egy nehezen olvasható, netalán érthetetlen szöveg teher minden játékos számára.

%----------------------------------------------------------------------------
\section{A dokumentum nyomdatechnikai kivitele}
%----------------------------------------------------------------------------
A dolgozatot A4-es fehér lapra nyomtatva, 2,5 centiméteres margóval (+1~cm kötésbeni), 11-12 pontos betűmérettel, talpas betűtípussal és másfeles sorközzel célszerű elkészíteni.


