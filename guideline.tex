%--------------------------------------------------------------------------------------
% Tételsor - 2016. tavasz
%--------------------------------------------------------------------------------------

\footnotesize
\begin{center}
\large
\textbf{\Large Vizsgatételek - 2016. tavaszi félév}\\
\end{center}

\vspace{6mm}
\textbf{Lineáris programozás}
\begin{enumerate}
\item	Az optimális hozzárendelés problémája, Egerváry algoritmusa.
\item	A lineáris programozás alapfeladata, kétváltozós feladat grafikus megoldása. Lineáris egyenlőtlenség-rendszer megoldása Fourier-Motzkin eliminációval.
\item	Farkas-lemma (két alakban). A lineáris program célfüggvénye felülről korlátosságának feltételei.
\item	A lineáris programozás dualitástétele (két alakban). A lineáris programozás alapfeladatának bonyolultsága (biz. nélkül).
\item	Egészértékű programozás: a feladat bonyolultsága, korlátozás és szétválasztás (Branch and Bound). Totálisan unimoduláris mátrix fogalma, példák. Egészértékű programozás totálisan unimoduláris együtthatómátrixszal (biz. nélkül).
\item	A lineáris és egészértékű programozás alkalmazása páros gráfokra és intervallumrendszerekre: Egerváry tétele, intervallumrendszerek egyenletes színezése.
\item	A lineáris és egészértékű programozás alkalmazása hálózati folyamproblémákra: a maximális folyam, a minimális költségű folyam és a többtermékes folyam feladatai, ezek hatékony megoldhatósága a tört-, illetve egészértékű esetben.
\end{enumerate}

\vspace{3mm}
\textbf{Matroidok}
\begin{enumerate}[resume]
\item	Matroid definíciója, alapfogalmak (bázis, rang, kör). Példák: lineáris matroid (mátrixmatroid), grafikus matroid, uniform matroid. A rangfüggvény szubmodularitása.
\item	Mohó algoritmus matroidon. Matroid megadása rangfüggvényével, bázisaival (biz. nélkül). Matroid duálisa, a duális matroid rangfüggvénye.
\item	Elhagyás és összehúzás. Matroidok direkt összege, összefüggősége. $T$ test felett reprezentálható matroid duálisának $T$ feletti reprezentálhatósága.
\item	Grafikus, kografikus, reguláris, bináris és lineáris matroid fogalma, ezek kapcsolata (ebből bizonyítás csak a grafikus és reguláris matroidok közötti kapcsolatra), példák. Fano-matroid, példa nemlineáris matroidra. Bináris, reguláris és grafikus matroidok jellemzése tiltott minorokkal: Tutte tételei (biz. nélkül).
\item	Matroidok összege. $k$-matroid metszet probléma, ennek bonyolultsága $k\geq3$ esetén.
\item	A $k$-matroid partíciós probléma, ennek algoritmikus megoldása. A 2-matroid-metszet feladat visszavezetése matroid partíciós problémára.
\item	$k$-polimatroid rangfüggvény fogalma. A 2-polimatroid-matching probléma, ennek bonyolultsága, Lovász tétele (biz. nélkül).
\end{enumerate}

\vspace{3mm}
\textbf{Közelítő és ütemezési algoritmusok}
\begin{enumerate}[resume]
\item	Polinomiális időben megoldható feladat fogalma, példák. Az NP, co-NP, NP-nehéz és NP-teljes problémaosztályok definíciója, viszonyaik, példák problémákra valamennyi osztályból. NP-nehéz feladatok polinomiális speciális esetei: algoritmus a maximális független ponthalmaz problémára és az élszínezési problémára páros gráfokon. Additív hibával közelítő algoritmusok speciális pont-, illetve élszínezési problémákra.
\item	A Hamilton-kör probléma visszavezetése a leghosszabb kör probléma additív közelítésére.\\ $k$-approximációs algoritmus fogalma, példák: két-két algoritmus a minimális lefogó ponthalmaz keresésére és a maximális páros részgráf keresésére. Minimális levelű, illetve maximális belső csúcsú feszítőfa keresése. Approximációs algoritmus az utóbbi feladatra (biz. nélkül).
\item	A minimális lefogó ponthalmaz visszavezetése az általános utazóügynök probléma $k$-approximációs megoldására. Közelítő algoritmusok a metrikus utazóügynök problémára, Christofides algoritmusa.
\item	A Hamilton-kör probléma visszavezetése az általános utazóügynök probléma $k$-approximációs megoldására. Közelítő algoritmusok a mtrikus utazóügynök problémára, Christofides algoritmusa.
\item	Teljesen polinomiális approximációs séma fogalma. A részösszeg probléma, bonyolultsága. Teljesen polinomiális approximációs séma a részösszeg problémára.
\item	Ütemezési feladatok típusai. Az $1|prec|C_{max}$ és az $1||\sum C_{j}$ feladat. Approximációs algoritmusok a $P||C_{max}$ feladatra: listás ütemezés tetszőleges sorrendben, éles példa tetszőleges számú gép esetére. Approximációs algoritmus a $P|prec|C_{max}$ feladatra (biz. nélkül), példák: az LPT sorrend, illetve a leghosszabb út szerinti ütemezés sem jobb, mint $(2-\frac {1} {m})$-approximáció. A $P|prec,p_{i}=1|C_{max}$ feladat, Hu algoritmusa (biz. nélkül).
\end{enumerate}

\vspace{3mm}
\textbf{Megbízható hálózatok tervezése}
\begin{enumerate}[resume]
\item	Globális és lokális élösszefüggőség és élösszefüggőségi szám fogalma, Menger irányítatlan gráfokra és élösszefüggőségre vonatkozó két tétele (biz. nélkül). $\lambda(G)$ meghatározása folyamatok segítségével négyzetes és lineáris számú folyamkereséssel.
\item	$\lambda(G)$ meghatározása összehúzások segítségével, Mader-tétele, Nagamochi és Ibaraki algoritmusa.
\item	Minimális méretű 2-élösszefüggő részgráfok keresése. A probléma NP-nehézsége, Khuller-Vishkin algoritmus (biz. nélkül).
\end{enumerate}

\vspace{3mm}
\textbf{Hálózatelméleti alkalmazások}
\begin{enumerate}[resume]
\item	Kirchoff tételei a klasszikus villamos hálózatok analízisére.
\item	Kirchoff eredményeinek általánosítása transzformátorokat vagy girátorokat is tartalmazó hálózatokra (biz. nélkül). Algoritmusok a feltételek ellenőrzésére.
\item	Kirchoff eredményeinek általánosítása: szükséges feltétel tetszőleges lineáris sok-kapukat is tartalmazó hálózatok egyértelmű megoldhatóságára. Villamos hálózatok duálisa.
\end{enumerate}

\vspace{3mm}
\textbf{Statikai alkalmazások}
\begin{enumerate}[resume]
\item	Rúdszerkezetek, merevségi mátrix, merevség, egyszerű rácsos tartók, Cremona-Maxwell diagramok.
\item	Minimális merev rúdszerkezetek általános helyzetben, Laman tétele (biz. nélkül), Lovász és Yemini tétele.
\item	Síkbeli négyzetrácsok és egyszintes épületek átlós merevítése.

\end{enumerate}

\vspace{6mm}

Felhasználható a következő oldaltól kezdődő \LaTeX-Diplomaterv sablon dokumentum tartalma. 

A diplomaterv szabványos méretű A4-es lapokra kerüljön. Az oldalak tükörmargóval készüljenek (mindenhol 2.5cm, baloldalon 1cm-es kötéssel). Az alapértelmezett betűkészlet a 12 pontos Times New Roman, másfeles sorközzel.

Minden oldalon - az első négy szerkezeti elem kivételével - szerepelnie kell az oldalszámnak.

A fejezeteket decimális beosztással kell ellátni. Az ábrákat a megfelelő helyre be kell illeszteni, fejezetenként decimális számmal és kifejező címmel kell ellátni. A fejezeteket decimális aláosztással számozzuk, maximálisan 3 aláosztás mélységben (pl. 2.3.4.1.). Az ábrákat, táblázatokat és képleteket célszerű fejezetenként külön számozni (pl. 2.4. ábra, 4.2 táblázat vagy képletnél (3.2)). A fejezetcímeket igazítsuk balra, a normál szövegnél viszont használjunk sorkiegyenlítést. Az ábrákat, táblázatokat és a hozzájuk tartozó címet igazítsuk középre. A cím a jelölt rész alatt helyezkedjen el.

A képeket lehetőleg rajzoló programmal készítsék el, az egyenleteket egyenlet-szerkesztő segítségével írják le (A \LaTeX~ehhez kézenfekvő megoldásokat nyújt).

Az irodalomjegyzék szövegközi hivatkozása történhet a Harvard-rendszerben (a szerző és az évszám megadásával) vagy sorszámozva. A teljes lista névsor szerinti sorrendben a szöveg végén szerepeljen (sorszámozott irodalmi hivatkozások esetén hivatkozási sorrendben). A szakirodalmi források címeit azonban mindig az eredeti nyelven kell megadni, esetleg zárójelben a fordítással. A listában szereplő valamennyi publikációra hivatkozni kell a szövegben (a \LaTeX-sablon a Bib\TeX~segítségével mindezt automatikusan kezeli). Minden publikáció a szerzők után a következő adatok szerepelnek: folyóirat cikkeknél a pontos cím, a folyóirat címe, évfolyam, szám, oldalszám tól-ig. A folyóirat címeket csak akkor rövidítsük, ha azok nagyon közismertek vagy nagyon hosszúak. Internet hivatkozások megadásakor fontos, hogy az elérési út előtt megadjuk az oldal tulajdonosát és tartalmát (mivel a link egy idő után akár elérhetetlenné is válhat), valamint az elérés időpontját.

\vspace{5mm}
Fontos:
\begin{itemize}
	\item A szakdolgozat készítő / diplomatervező nyilatkozata (a jelen sablonban szereplő szövegtartalommal) kötelező előírás Karunkon ennek hiányában a szakdolgozat/diplomaterv nem bírálható és nem védhető !
	\item Mind a dolgozat, mind a melléklet maximálisan 15 MB méretű lehet !
\end{itemize}

\vspace{5mm}
\begin{center}
Jó munkát, sikeres szakdolgozat készítést ill. diplomatervezést kívánunk !
\end{center}

\normalsize
