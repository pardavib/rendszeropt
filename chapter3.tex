
\setcounter{chapter}{3}
%----------------------------------------------------------------------------
\chapter*{3. Tétel}
%----------------------------------------------------------------------------

\textbf{Témakörök:} Farkas-lemma (két alakban). A lineáris program célfüggvénye felülről korlátosságának feltételei.

\noindent\hrulefill

\section*{Farkas-lemma 1.}
Tetszőleges $A$, $b$ esetén az alábbi rendszerek közül pontosan az egyiknek van megoldása:
\begin{itemize}
\item (1) $Ax\leq b$
\item (2) $yA=0$, $y\geq 0$, $yb<0$
\end{itemize}

\section*{Farkas-lemma 2.}
Tetszőleges $A$, $b$ esetén az alábbi rendszerek közül pontosan az egyiknek van megoldása:
\begin{itemize}
\item (1) $Ax=b$, $x\geq 0$
\item (2) $yA\geq 0$, $yb<0$
\end{itemize}

\textbf{Megjegyzés:} a tétel (1) állítása felfogható egyenlőtlenség rendszerként is: $Ax\leq b$, $(-A)x\leq (-b)$, $(-E)x\leq 0$. Erre is alkalmazható a Farkas-lemma 1. alakja.

\section*{Következmény}
Tetszőleges $A$, $b$ esetén az alábbi rendszerek közül pontosan az egyiknek van megoldása:
\begin{itemize}
\item (1) $Ax=b$
\item (2) $yA=0$, $yb\neq 0$
\end{itemize}

\section*{Célfüggvény korlátossága}
Tegyük fel, hogy $Ax\leq b$ megoldható, $c$ tetszőleges adott vektor. Ekkor az alábbi állítások ekvivalensek:
\begin{itemize}
\item (1) az $Ax\leq b$ megoldáshalmazán $cx$ felülről korlátos,
\item (2) nincs megoldása az $Az\leq 0$, $cz>0$ rendszernek,
\item (3) van megoldása az $yA=c$, $y\geq 0$ rendszernek.
\end{itemize}