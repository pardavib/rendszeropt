
\setcounter{chapter}{9}
%----------------------------------------------------------------------------
\chapter*{9. Tétel}
%----------------------------------------------------------------------------

\textbf{Témakörök:} Mohó algoritmus matroidon. Matroid megadása rangfüggvényével, bázisaival (biz. nélkül). Matroid duálisa, a duális matroid rangfüggvénye.

\noindent\hrulefill

\section*{Mohó algoritmus (röviden)}

Legyen $M=(E,F)$ matroid, $w:E\rightarrow \mathbb{R}_{+}$ nemnegatív súlyfüggvény. Keressük a maximális összsúlyú független halmazt, azaz: $\max\limits_{X\in F} \sum\limits_{e\in X} w(e)$.\\
A mohó algoritmus tetszőleges matroidra és súlyfüggvényre optimális (maximális összsúlyú) megoldást ad.

\section*{Matroid megadása}

Függetlenségi axiómákkal (lásd: előző tétel).

\section*{Megadás bázisokkal}
\begin{itemize}
\item (B1) $B\neq\emptyset$,
\item (B2) $|X_{1}| = |X_{2}|$ minden $X_{1},X_{2}\in B$-re,
\item (B3) ha $X_{1},X_{2}\in B$ és $e_{1}\in X_{1}$, akkor létezik olyan $e_{2}\in X_{2}$, melyre $X_{1}-e_{1}+e_{2}\in B$.
\end{itemize}
Fordítva: ha $(E,B)$ egy halmazrendszer a (B1), (B2) és (B3) tulajdonságokkal, akkor $M=(E,F)$ matroidot alkot, ahol: $F=\lbrace H:H\subseteq B\rbrace$ valamely $B\in B$-re.

\section*{Megadás rangfüggvénnyel}
Lásd: korábban

\section*{Egyéb fogalmak}
\begin{defn} [Lezárt]
$(E,F)$ matroidban egy $X\subseteq E$ halmaz lezártja egy maximális olyan halmaz, mely tartalmazza $X$-et és rangja megegyezik $X$ rangjával. Jele: $\overline{X}$.
\end{defn}

\begin{defn} [Zárt halmaz]
egy $X$ halmaz zárt, ha $X=\overline{X}$.
\end{defn}

\begin{defn} [Izomorfia]
Két matroid izomorf, ha létezik olyan bijekció a két alaphalmaz között, mely független halmazt független halmazba visz. Jele: $M\equiv M^{'}$.
\end{defn}

\section*{Duális}
$M=(E,F)$, és $M$ duálisának alaphalmaza legyen $E$, és egy $X\subseteq E$ halmaz akkor legyen az új halmazrendszer eleme, ha $E-X$ tartalmaz $M$-beli bázist. Ezt a halmazt jelöljük $F^{*}$-al.

\begin{defn}
$M=(E,F)$ matroid bázisai $B=\lbrace B_{1},B_{2},\cdots ,E-B_{n}\rbrace$. Ebből már adódik $F^{*}$.
\end{defn}

\begin{theo} [Duális matroid tétel] Az $M^{*} =(E,F^{*})$ matroid.\\
FYI: $(M^{*})^{*}\equiv M$
\end{theo}

\subsection*{Példa}
$M=U_{5,2}$:
\begin{itemize}
\item $M$-ben minden legfeljebb 2-elemű halmaz független,
\item a duálisban azon halmazok függetlenek, amelyek komplementerei tartalmaznak $M$-beli bázist, azaz 2-elemű halmazt,
\item ezek a legfeljebb 3-elemű halmazok, tehát $M^{*}=U_{5,3}$.
\end{itemize}

\subsection*{Duális rangfüggvény}
$r^{*}(X)=|X|-r(E)+r(E-X)$