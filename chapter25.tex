
\setcounter{chapter}{25}
%----------------------------------------------------------------------------
\chapter*{25. Tétel}
%----------------------------------------------------------------------------

\textbf{Témakörök:} Kirchoff eredményeinek általánosítása transzformátorokat vagy girátorokat is tartalmazó hálózatokra (biz. nélkül). Algoritmusok a feltételek ellenőrzésére.

\noindent\hrulefill

\section*{További áramköri elemek}
Transzformátor:
\begin{center}
$U_{2}=k*U_{1}$\\
$I_{1}=-k*I_{2}$
\end{center}
Girátor:
\begin{center}
$U_{1}=R*I_{2}$\\
$U_{2}=-R*I_{1}$
\end{center}

\begin{theo} [1]
$U,I,R$ és ideális transzformátorokból álló hálózat egyértelműen megoldható, ha $\exists F$ fa, melyre:
\begin{itemize}
\item[•] $E_{U}\subseteq F$ és $E_{I}\cap F=\emptyset$,
\item[•] $|F\cap\lbrace e_{k},f_{k}\rbrace |=1$ $\forall k$-ra ($\lbrace e_{k},f_{k}\rbrace$ transzformátor élpárok).
\end{itemize}
\end{theo}

\begin{theo} [2]
$U,I,R$ és girátorokból álló hálózat egyértelműen megoldható, ha $\exists F$ fa, melyre:
\begin{itemize}
\item[•] $E_{U}\subseteq F$ és $E_{I}\cap F=\emptyset$,
\item[•] $|F\cap\lbrace e_{k},f_{k}\rbrace |\neq 1$ $\forall k$-ra ($\lbrace e_{k},f_{k}\rbrace$ girátor élpárok).
\end{itemize}
\end{theo}

\subsection*{Kombinálva}
A fentieknek eleget tevő fákra határozzuk meg:\\
$max\lbrace C_{F}=|E_{C}\cap F|+|E_{L}-F|\rbrace$ a differenciálegyenlet rendje.\\
\newline
Ebben a fában:\\
$X_{C}=E_{L}\cap F$\\
$X_{L}=E_{L}-F$\\
ekkor $X_{L}$ tekercsek és $X_{C}$ kondenzátorok kezdeti értékei függetlenül előírhatóak.

\section*{Feltételek ellenőrzése}
\begin{itemize}
\item[a)] Korábbi tételek: szélességi kereséssel vagy mohó-algoritmussal.
\item[b)] (1) Tétel: matroidmetszet-algoritmussal:
\begin{itemize}
\item $E_{U}$ körmentes, $E_{I}$ vágásmentes $\rightarrow$ OK,
\item körmatroidban $E_{U}$-t húzzuk össze, $E_{I}$-t hagyjuk el ($M^{'}$),
\item $M^{'}$ $S$ alaphalmaza $a+2b$ elemű, $r(M^{'})=r$,
\item $U(S,F^{'}):$ $X\in F^{'}$, ha $\forall\lbrace e_{k},f_{k}\rbrace$ párból legfeljebb egyet, és legfeljebb $r-b$ ellenállást tartalmaz,
\item a feltétel teljesül, ha $M^{'}$-nek és $U$-nak van közös bázisa.
\end{itemize}
\item[c)] (2) Tétel: matroidpárosítási-algoritmussal:
\begin{itemize}
\item polinomidőben ellenőrizhető,
\item olyan matroidra alkalmazzuk, melynek ismert a valósak teste feletti koordinátázása.
\end{itemize}
\end{itemize}


