
\setcounter{chapter}{14}
%----------------------------------------------------------------------------
\chapter*{14. Tétel}
%----------------------------------------------------------------------------

\textbf{Témakörök:} $k$-polimatroid rangfüggvény fogalma. A 2-polimatroid-matching probléma, ennek bonyolultsága, Lovász tétele (biz. nélkül).

\noindent\hrulefill

\section*{k-polimatroid rangfüggvény}
$f:2^{E}\rightarrow \mathbb{N}$ egy k-polimatroid rangfüggvény, ha teljesülnek rá az alábbi axiómák:
\begin{itemize}
\item (1) $r(\emptyset )=0$,
\item (2) $r(\lbrace x\rbrace)\leq k$, $\forall x\in E$ elemre ,
\item (3) $X\subseteq Y \Rightarrow r(X)\leq r(Y)$,
\item (4) $r(X)+r(Y)\geq r(X\cap Y)+r(X\cup Y)$.
\end{itemize}
Általánosítása a matroid rangfüggvénynek, értsd:\\
$k=1$: speciális eset: $r(\lbrace x\rbrace )\leq 1$, ekkor $R$ egy matroid rangfüggvénye.\\
\textbf{Megjegyzés:} $r(\lbrace x\rbrace )\leq k$-val ekvivalens az $r(X)\leq k|X|$, illetve az $r(X\cap Y)\leq r(X)+k|Y|$ axióma.

\section*{k-polimatroid matching probléma (k-PMM)}
\begin{defn} $X\subseteq E$ k-polimatroid matching, ha $r(X)=k|X|$ egyenlőség fennáll.\end{defn}
\begin{defn} k-polimatroid matching probléma:\\
Adott: $r$ és $t\in \mathbb{N}$\\
Kérdés: van-e legalább $T$ elemű k-polimatroid matching?
\end{defn}

\subsection*{Speciális esetek}
\begin{itemize}
\item[•] Input: tetszőleges $G$ gráf, $t\in \mathbb{N}$. Kérdés: $\nu(G)\geq t$?\\
2-PMM-ként megfogalmazva: $r(X)=|X$ által lefedett pontok halmaza $|\leq 2|X|$\\
A 2-matching épp a közönséges párosításnak felel meg (innen az elnevezés).
\item[•] Input: két maroid, $t\in \mathbb{N}$. Kérdés: létezik-e $X\subseteq E$, $|X|\geq t$, $X\in F_{1}\cap F_{2}$ (matroid metszet probléma)?\\
2-PMM-ként megfogalmazva: $f(X)=r_{1}(X)+r_{2}(X)\leq 2|X|$
\item[•] Az utolsó két probléma közös speciális esete: páros gráfban $\nu (G)\geq t$?\\
Megoldás: a 2. probléma leképezése a 3.-ra:\\
$M_{1}$ grafikus matroidban $e_{1}$ és $e_{2}$ párhuzamos élek, ha $e_{1}$-nek és $e_{2}$-nek felül van egy közös pontja. $M_{2}$-t hasonlóan értelmezzük a páros gráf alsó ponthalmazán.
\end{itemize}

\subsection*{Bonyolultság}
\begin{itemize}
\item $k\geq 3$ eset: NP-nehéz, mert speciális esetként tartalmazza a k-MMP-t.
\item $k=2$ eset: "matroidpárosítási probléma", speciális esetként tartalmazza 2-MMP-t.
\end{itemize}

\begin{theo}
A matroidpárosítási probléma (2-PMM) teljes általánosságban nem oldható meg polinomidőben. A teljes általánosságban kifejezés a függvény megadási módjára vonatkozik: azt jelenti, hogy bármely $X\subseteq E$ részhalmazra egységnyi idő alatt megtudhatjuk $r(X)$ értékét, de ettől eltekintve a 2-polimatroid rangfüggvényről semmit sem tudunk.
\end{theo}


\section*{Lovász László tétele}
"Legfontosabb speciális eset."
\begin{theo}
Vegyünk egy $k\times 2n$ méretű valós $M$ mátrixot, oszlopai legyenek rendre $M=(a_{1},b_{1},a_{2},b_{2},\cdots ,a_{n},b_{b})$ majd definiáljunk az $I=\lbrace 1,2,\cdots ,n\rbrace$ indexhalmazon egy $r$ függvényt úgy, hogy $X\subseteq I$ esetén legyen $r(X)$ az $\cup_{i\in X}\lbrace a_{i},b_{i}\rbrace$ vektorhalmaz által kifeszített altér dimenziója. Könnyű látni, hogy ilyenkor $r$ egy 2-polimatroid rangfüggvény.
\end{theo}

\begin{theo}
A matroidpárosítási probléma polinomidőben megoldható, ha a 2-polimatroid rangfüggvény egy adott valós elemű $M$ mátrixból a fent leírt módon nyerhető.
\end{theo}