
\setcounter{chapter}{8}
%----------------------------------------------------------------------------
\chapter*{8. Tétel}
%----------------------------------------------------------------------------

\textbf{Témakörök:} Matroid definíciója, alapfogalmak (bázis, rang, kör). Példák: lineáris matroid (mátrixmatroid), grafikus matroid, uniform matroid. A rangfüggvény szubmodularitása.

\noindent\hrulefill

\section*{Mohó algoritmus}

Olyan algoritmus, amely a következők szerint működik:
\begin{itemize}
\item Adott: $E$,$F$ és $w$, ahol $F$ nem üres halmazrendszer, leszálló, $w$ nemnegatív.
\item Kiindulás: $0\in F$ megengedett megoldásból.
\end{itemize}

Ha a pillanatnyi megoldás $X\subset E$, és létezik olyan elem, melyet $X$-hez hozzávéve az új halmaz is $F$-ben van, akkor legyen $e$ olyan elem, melyre: \\
$w(e)=max\lbrace w(e^{'}):e^{'}\in E-X:X+e^{'}\in F\rbrace$ és legyen az új halmaz $X+e$. Ha nincs hozzávehető elem, akkor készen vagyunk.

\section*{Matroid}
\begin{defn}
Egy $E$ alaphalmazon értelmezett, nemüres, leszálló halmazrendszer metroid, ha tetszőleges nemnegatív súlyfüggvényre a mohó algoritmus optimális - maximális súlyú - megoldást ad.
\end{defn}

\section*{Fügetlenségi aximómák}
Legyen $F$ olyan halmazrendszer $E$-n, melyre teljesül az alábbi 2 feltétel:
\begin{itemize}
	\item (F1) $\emptyset \in F$,
	\item (F1) ha $Y\subseteq X$ és $X\in F$, akkor $Y\in F$.
\end{itemize}
Ekkor $M=(E,F)$ akkor és csak akkor matroid, ha teljesül az alábbi:
\begin{itemize}
	\item (F3) Ha $X,Y\in F$ és $|X|>|Y|$, akkor létezik olyan $x\in X-Y$, melyre $Y+x\in F$.
\end{itemize}

\section*{Alapfogalmak}
\begin{itemize}
\item[•] \textbf{Független halmazok:} $M=(E,F)$ matroidban az alaphalmaz $F$-hez tartozó részhalmazai.
\item[•] \textbf{Összefüggő halmaz:} ha $X\subseteq E$ és $X\notin F$, akkor $X$ összefüggő.
\item[•] \textbf{Bázisok:} a maximális (nem bővíthető) független halmazok.
\item[•] \textbf{Körök:} a tartalmazásra nézve minimális összefüggő halmazok.
\item[•] \textbf{Hurok:} az egyelemű kör.
\item[•] \textbf{Rang:} $X\subseteq E$ halmaz ranja $r(X)$ egy $X$-beli maxfüggetlen halmaz mérete.
\item[•] \textbf{Rangfüggvény:} $r: 2^{E} \rightarrow \mathbb{Z}$ függvény (a matroid rangja $t$, ha $r(E)=t$).
\end{itemize}

\begin{lem}
$M=(E,F)$ matroid, $A\subseteq E$. Ha $X_{1}$ és $X_{2}$ maxfüggetlen halmazok $A$-ban, akkor $|X_{1}|=|X_{2}|$.
\end{lem}

\section*{Példák}
\begin{itemize}
\item[•] \textbf{Grafikus matroid:} $G$ gráf által indukált matroid, melynek független halmazai a $G$-beli erdők. Jele: $M(G)$, másik neve: körmatroid.
\item[•] \textbf{Lineáris matroid:} valamely mátrix oszlopvektorai által indukált matroid. Másik neve: mátrixmatroid.
\item[•] \textbf{Uniform matroid:} $F$ az $n$ elemű $E$ alaphalmaz összes legfeljebb $k$ elemű halmazából áll ($0\leq k\leq n$). Ekkor $(E,F)$ matroid. Jele: $U_{n,k}$, $U_{n,n}$ a teljes vagy szabad matroid, $U_{n,0}$ a triviális matroid.
\end{itemize}

\begin{theo}
Egy uniform matroid grafikus, ha $U_{n,0}$, $U_{n,1}$, $U_{n,n-1}$ vagy $U_{n,n}$ alakú.
\end{theo}

\section*{Rangfüggvény szubmodularitása}
Legyen $R$ egy matroid rangfüggvénye, ekkor:
\begin{itemize}
\item (R1) $r(\emptyset )=0$,
\item (R2) $r(X)\leq |X|$, minden $X\subseteq E$-re,
\item (R3) $r(Y)\leq r(X)$, ha $Y\subseteq X$,
\item (R4) $r(X)+r(Y)\geq r(X\cap Y) + r(X\cup Y)$ minden $X,Y\subseteq E$ halmazpárra.
\end{itemize}
Fordítva: ha $r$ egy egészértékű függvény $E$ részhalmazain, melyre (R1)-(R4) teljesülnek, akkor $r$ egy $M=(E,F)$ matroid rangfüggvénye, ahol: $F=\lbrace H: r(H)=|H|\rbrace$
